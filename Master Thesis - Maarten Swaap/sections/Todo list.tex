Done - Verbeteren appendix indeling tabellen en nummering
Done - External validity beschrijven
Done 15-01 - 
8 Threats to validity - Internal validity schrijven
2.2.2 Domein van identiteitsfraude uitbreiden met 'Identiteitsdiefstal' en 'Ongewenst combineren datasets'
2.2.3 Quantitatieve identiteitsfraude uitbreiden met onderzoek van 'Determinants of reporting cybercrime_ A
comparison between identity theft, consumer fraud, and hacking'
2.2.5 (of 2.2.4 lager plaatsen) Operationele probleemstelling waar oplossingen betrekking op hebben

Done 22-01
Appendix - Concerns uitbreiden met concerns van laatste interview
Transcripties annonimiseren
Transcripties opnemen in replication package

Done 28-01
5.1 ZKP omschrijving en toepassing verbeteren
2.2.6 Rol BRP beschrijven icm probleemstelling
4 Link aanbrengen tussen concrete problemen van identiteitsdiefstal en geschetste oplossingen

Done 29-01
6.1 Nakijken op correctheid
6.2 Nakijken op correctheid (bijv. focus groep weg laten).
6.3 Obtained results schrijven

\todo
29-01
Transcripties voorleggen aan geïnterviewden t.b.v. NDA en validatie conclusies/uitkomsten
3 Related work verbeteren, is nog niet van het juiste niveau
3 Related work uitbreiden met nog minimaal één artikel

2-02
5.3 oplossingen koppelen aan juist Trade-offs en concerns verbeteren
Uitbreiden coding file met koppeling naar QA en ASR.
Optioneel: 4 overview uitbreiden met ASR Security/privacy als impliciet ASR

7 Discussion RQ1 aanscherpen
7 Discussion RQ2 schrijven
7 Discussion RQ3 schrijven
9.1 Probleem aan oplossingen linken
9.2 Future research schrijven (o.a. met opmerking uit 2.2.3)
--> Niet makkelijk te kwantificeren wat het effect van identiteitsdiefstal is, dus ook niet eenvoudig om een oplossing te kwantificeren
Appendix - Business goals aanscherpen om link te leggen met probleem en QA/ASR
References - check of de referenties compleet zijn (ingevoerd, danwel overgenomen in referentie bijlage)
Abstract vullen
Presentatie voor verdediging
Presentatie aan Michiel