\chapter{Related Work}\label{s:related}

\section{An ISO 25010 Based Quality Model for ERP Systems}
The research objective by Peters \etal \cite{Peters2020AnI2} is to provide guidance in implementing ERP systems in higher education institutions (HEI). The quality attributes of the ISO 25010 standard \cite{ISO:25010:2011} are specifically describing characteristics necessary for selecting an appropriate ERP system within HEIs. The article describes the application of each of the eight software quality characteristics (referred to as factors) of the ISO 25010 standard and argue the implementation of an ERP justifies adding three new sub-characteristics (or sub-factors) suitable for ERP selection within HEIs. This article adds Supportability, Searchability and Archivability as sub-factors and plots the ISO 25010 standard on a specific type of application. This article maintains a high-level and generic description of Quality attributes and sub-factors. Specifying characteristics could be relevant for implementation of systems to improve privacy of citizens. However, this research differs from the method in this article because ISO 25010 is deemed sufficient combined with methodology of Bass \etal \cite{Bass2015SoftwareAI} to specify Business Goals and Concerns. 

\section{Consensus Building when Comparing Software Architectures}
Research by Svahnberg and Wohlin \cite{Svahnberg2002ConsensusBW} provides a method to create frameworks and a comparison model to assess suitability of software architectures within those frameworks. The applied ISO 9126 standard \cite{ISO9126} is a predecessor of ISO 25010 \cite{ISO:25010:2011}. Quantification of each quality attribute in each framework assessment is based on input of participants in questionnaires and based on identified points of disagreement. This method can be relevant for this research because this method could provide a way to quantify selection of different possible architectures based on quality attributes. It can be reproduced based on ISO 25010 quality attributes extracted from this research. However, due to restricted amount of time it's not possible to execute the method described.

\section{A guideline for software architecture selection based on ISO 25010 quality related characteristics}
Literature research by Haoues \etal \cite{Haoues2017AGF} claim it is possible to select appropriate software architecture by creating standardized selection guidelines utilizing Software product quality model like ISO 25010 \cite{ISO:25010:2011}. The research concludes Functional Suitability, Performance efficiency, Usability and Compatibility are the main drivers to select a type of Software Architecture. This is done by comparing the eight main quality characteristics of ISO 25010 and weighing each attribute withing different type of architecture. The claim is based on extensive literature study and is relevant for this study because it can help predict possible suitable Software Architectures. However, due to restricted amount of time it's not possible to execute the method described. 

\section{A methodology to evaluate the safety-based with ISO 25010:2011} \todo{deze alinea nakijken}
Research by {\'A}lvarez \etal \cite{Mexlvarez2021AMT} based on structured questionnaires assesses web portal security and categorized this in five main Quality attributes originating from the ISO 25010 standard \cite{ISO:25010:2011}. The researched sample shows a 70\% negative score on these Quality attributes proving weakness in security of these systems. The Quality attributes of Confidentiality, Authenticity and Accountability are represented in my research and considered an ASR. Similarity to other research which examines security of information transfer is important to substantiation these quality attributes can be deemed relevant. Two other Quality attributes mentioned are Non-repudiation and Integrity. These Quality attributes focus on preventing unauthorized access and logging of actions in a system. Not represented in the outcome of my research does make these Quality Attributes obsolete. However, it could be argued when applying patterns like zero-knowledge proof breach in authentication should not result in privacy impact.