\chapter{Related Work}\label{s:related}

%\todo{
%Describe here scientific papers similar to your experiment, both in terms of %goal and methodology.  One paragraph for each paper (we expect about 5-8 papers %to be discussed). Each paragraph contains: (i) a brief description of the %related paper and (ii) a black-on-white description about how your work differs %from the related paper. You may place this section immediately after the %Background section, if necessary.
%}

\section*{An ISO 25010 Based Quality Model for ERP Systems}
The objective of this research by Peters \etal \cite{Peters2020AnI2} is to provide a guide in implementing ERP systems in higher education institutions (HEI). The ISO 25010 standard is used as a base to specify the quality attributes necessary for selecting an appropriate ERP system for Higher Education institutions. It describes the application of each of the eight software quality characteristics also referred to as factors of the ISO 25010 standard and argue the implementation of an ERP justifies adding three new sub-characteristics (or sub-factors) suitable for ERP selection within HEIs. It adds Supportability, Searchability and Archivability as sub-factors. This research plots the ISO 25010 standard on a specific type of application and therefore can be valuable to consult when selecting applications in another area of expertise, like privacy. However, this research stays at a high-level and general description of Quality attributes and sub-factors. While, this research obtains to specify characteristics deemed relevant when implementing systems to improve privacy of citizens. Following methodology of Bass \etal \cite{Bass2015SoftwareAI} to specify Business Goals and Concerns. 

\section*{Consensus Building when Comparing Software Architectures}
Research by Svahnberg and Wohlin \cite{Svahnberg2002ConsensusBW} provides a method to create frameworks and a comparison model to compare these frameworks, in order to select the software architectures deemed most qualified. Research is based on the ISO 9126 standard \cite{ISO9126}, which is a predecessor of ISO 25010. Creating a quantification for each quality attribute per Framework assessment is based on input of participants in questionnaires. Quantification is based on identified points of disagreement. It is interesting and relevant for this research, because it provides a way to quantify selection of different possible architectures based on quality attributes. It can be reproduced based on ISO 25010 and the quality attributes extracted from this research. However, due to restricted amount of time it's not possible to execute the method described.

\section*{A guideline for software architecture selection based on ISO 25010 quality related characteristics}
Literature research by Haoues \etal \cite{Haoues2017AGF} claim it's possible to apply a Software Architecture selection guideline by using a Software product quality model like ISO 25010 to select an appropriate type of software architecture. In the research a comparison between the eight main quality characteristics of ISO 25010 has been done in relation to different type of architectures and conclude Functional Suitability, Performance efficiency, Usability and Compatibility are the main drivers to select a type of Software Architecture. The claim is based on extensive literature study and is relevant for this study because it can help predict possible suitable Software Architectures. However, due to restricted amount of time it's not possible to execute the method described. 

\section*{A methodology to evaluate the safety-based witch ISO 25010:2011}
Research by {\'A}lvarez \etal \cite{Mexlvarez2021AMT} based shows an assessment of web portal security based on structured questions categorized in five main Quality attributes originating from ISO 25010. The researched sample shows a 70\% negative score on these Quality attributes, proving weakness in these systems. The Quality attributes of Confidentiality, Authenticity and Accountability are represented in my research and considered an ASR. This is an important similarity to prove these are deemed relevant in other researches. Two other Quality attributes mentioned are Non-repudiation and Integrity. These Quality attributes focus on preventing unauthorized access and logging of actions in a system. Not represented in the outcome of my research does make these Quality Attributes obsolete. However, it could be argued when applying patterns like zero-knowledge proof breach in authentication should not result in privacy impact. 
