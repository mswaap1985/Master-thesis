\chapter{Related Work}\label{s:related}


\todo{
Describe here scientific papers similar to your experiment, both in terms of goal and methodology.  One paragraph for each paper (we expect about 5-8 papers to be discussed). Each paragraph contains: (i) a brief description of the related paper and (ii) a black-on-white description about how your work differs from the related paper. You may place this section immediately after the Background section, if necessary.
}

\section*{An ISO 25010 Based Quality Model for ERP Systems}
Peters \etal \cite{Peters2020AnI2} provide a guide in implementing ERP systems in higher education institutions (HEI). This research adapts the ISO 25010 standard to specify the quality attributes necessary for selecting an appropriate ERP system for Higher Education institutions. The research describes the application of each of the eight (8) software quality characteristic, also called factors and justifies three (3) new sub-characteristics (or sub-factors) suitable for ERP selection within HEIs. It adds Supportability, Searchability and Archivability as sub-factors. The paper is relevant for this research, because it plots the ISO 25010 on a single type of application. However, this research will not plot every characteristic and sub-characteristic but will point out relevant characteristics. Or Quality Attributes mentioned by Bass \etal \cite{Bass2015SoftwareAI}.

\section*{Consensus Building when Comparing Software Architectures}
Svahnberg and Wohlin \cite{Svahnberg2002ConsensusBW} provide a comparison method on how to produce a comparison of architectures based on quality attributes of the ISO 9126 standard \cite{ISO9126}. The paper explains a method on how to quantify the impact of different architecture types on the assessed quality attributes. It's relevant for my research, because it provides a way to quantify selection of different possible architectures, based on quality attributes. It can be reproduced, based on ISO 25010 and the quality attributes extracted from this research. However, due to restricted amount of time it's not possible to execute the method described. 

\section*{Methodology of Evaluating the Sufficiency of Information for Software Quality Assessment According to ISO 25010}
Tetiana Hovorushchenko \cite{Hovorushchenko2018MethodologyOE}

\section*{A guideline for software architecture selection based on ISO 25010 quality related characteristics}
Haoues \etal \cite{Haoues2017AGF} claim it's possible to apply a Software Architecture selection guideline to select an appropriate type of software architecture. They compare relations of the eight (8) main quality characteristics with each other and conclude Functional Suitability, Performance efficiency, Usability and Compatibility are the main drivers to select a type of Software Architecture. The claim is based on extensive literature study and is relevant for this study because it can help predict possible suitable Software Architectures. However, due to restricted amount of time it's not possible to execute the method described. 