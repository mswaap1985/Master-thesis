\chapter{Threats To Validity}\label{sec:threats}
Threats to validity of the experiment, according to the classification framework by Wohlin \etal \cite{wohlin12}.
%https://link-springer-com.vu-nl.idm.oclc.org/content/pdf/10.1007%2F978-3-642-29044-2.pdf

\section{Internal Validity}
This research provides a set of architecturally significant requirements, tactics and patterns. Combined with an interpretation how to embed these tactics and patterns.
This set can be considered as a starting point and viable selection. However, it will not rule out these tactics and patterns are already in place or other tactics and patterns are better suitable in some situations. If the interviews would be conducted with different subjects in other government or non-governmental organisations it can be assumed the set of variables will be extended or even considered inexhaustible.

\section{External Validity}
A restricted amount of time and available resources resulted in a limited sample of selected interview subjects. The outcomes of the interviews can be considered as a pilot. A starting point for further research and actions within the organisation. 
However, the interviewees are experts in the domain of personal identity within the Dutch government. Consulted interdepartmental by other government institutions when it comes to the design and content of the identity system. A source of truth and historical design decisions for different ministries and government departments. The directing architect is involved as peer reviewer for government standards. All subjects have more than 10 years experience at the National office for identity data.  

\section{Construct Validity}
Within this qualitative research semi-structured interviews are used to collect data. Firstly, this method is chosen because the subjects are experts in their fields of expertise, namely personal identity. Limiting them to a questionnaire or structured interview would not provide unexpected and new information or insights in their field of expertise or a correct understanding of the operational problem. Secondly, the use data gathering techniques like unstructured interviews could result in an unlimited stream of information and history. Blandford \etal \cite{Blandford2016QualitativeHR} define that semi-structured interviews “inevitably bring in the interests of the researcher as well as the participant.”. 
\par
It’s needed to acknowledge the bias introduced by the researcher. Blandford \etal \cite{Blandford2016QualitativeHR} define the shaping of data gathering by a researcher as follows: “The researcher is shaping the conversation and the data that is gathered, and the extent of that shaping should be recognized and reported transparently and unapologetically.” The relationship between researcher and subjects is as colleagues within a Dutch government body.

\section{Reliability}
To test reliability it has been made possible to verify or replicate this research by providing a replication package. Replication package is available on GitHub \url{https://github.com/mswaap1985/Master-thesis}.
The implementation sections covers one pattern and a set tactics. It can be argued there is a more extensive set of patterns and tactics that can be applied.
Trade-offs and possible gaps in these solutions are composed by logical reasoning of researcher. Discussion is needed to validate the items before implementing a solution. 