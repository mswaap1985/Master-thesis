\chapter{Discussion}\label{s:discussion}
In the planning phase of this research the research questions were logically ordered from problem to possible solution. However, in practice it made more sense to firstly answer RQ3, because this provided a broader representation of the problems that could apply. 

This research leaves implications in current law out of scope, because this in not the field of expertise of researcher. Implementations based on this work need to be evaluated for a legal base.

\begin{quote}\emph{RQ1: Which forms of identity fraud can be identified that could be mitigated with usage of encryption methodologies?}\end{quote}
In practice, preventing identity fraud is only one of the many business goals to take into account. Because of it's importance it is categorized as an Architectural Significant Requirement and categorized as part of the Confidentiality Quality attribute of ISO 25010. One could argue it's not quantifiable what kind of real world problems are mitigated with each Quality Attribute scenario. However, with a rationale and assumptions covered for each QA scenario, this should provide grip evaluating or designing a system. For future work it can be relevant to investigate the impact of mitigating measures and what will drive a citizen to use, or not use, a government system.
\begin{quote}\emph{RQ2: Which forms of encryption methodologies are available and already implemented or being implemented?}\end{quote}

\begin{quote}\emph{RQ3: Which requirements are significantly relevant to assess technical suitability and trade offs of these technologies for broader usage within the Dutch government?}\end{quote}

%\todo{
%Here you put your results in context (possibly grouped by research question). %Usually, this section focuses on analyzing the
%implications of the proposed work for current and future research and for %practitioners.
%}