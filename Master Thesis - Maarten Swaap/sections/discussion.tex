\chapter{Discussion}\label{s:discussion}
Implications on law are left out of scope of this research, because of time and capacity restrictions. Future implementations based on this work need to be assessed from a legal point of view.

\begin{quote}\emph{RQ1: What is the definition of identity fraud and which concrete problems can be identified?}\end{quote}
In the design stage of this research the definition of identity fraud was considered abstract and clear to start working on solutions. However, during the research it became clear this was still a relevant subject which needed a clear definition and quantification. Also, refining the problem to a size it could be operational took more time than originally assumed in the design phase. An addition to this research could be a quantified method to assess impact on identity fraud when a certain solution is implemented. Possibly resulting in conclusions containing correlation and no causality. Proving an event not happening would also assume one could have prevented it already leading to a paradoxical conclusion.

\begin{quote}\emph{RQ2: Which requirements are significantly relevant to assess technical suitability and trade offs of these technologies for broader usage within the Dutch government?}\end{quote}
In practice, preventing identity fraud is only one of the many business goals to take into account. Also, it could depend on the area of expertise of the beholder how to answer this question. Confidentiality is an Architectural Significant Requirement and a Quality attribute of ISO 25010, but not the only Quality attribute that is relevant to consider when designing a system. Also, one could argue it's not quantifiable what kind of real world problems are mitigated with each implemented pattern of tactic in Chapter \ref{s:Implementation}. However, logical reasoning should provide grip evaluating or designing a system. For future work it can be relevant to investigate the impact of mitigating measures and what will drive a citizen to use, or not use, a government system.

\begin{quote}\emph{RQ3: Which architectural tactics and patterns are available and can be used to mitigate identity fraud?}\end{quote}
The selected patterns and tactics are based on the input of consulted experts and interviewees. However, the possible patterns, tactics and matching solutions are inexhaustible and depend on the problem at hand. Two of the concerns not solved in Chapter \ref{s:Implementation} are C-01 "Knowledge of systems an what is legally allowed" and C-02 "Knowledge on encryption and standards is scarce". Both concerns refer to a knowledge problem which could result in problems during implementation or life-cycle. Section \ref{Implications} will state more explicitly  Implications for future research on this topic.

%\todo{
%Here you put your results in context (possibly grouped by research question). %Usually, this section focuses on analyzing the
%implications of the proposed work for current and future research and for %practitioners.
%}
