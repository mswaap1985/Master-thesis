\chapter{Discussion}\label{s:discussion}
Implications on law are left out of scope of this research, because of time and capacity restrictions. Future implementations based on this work need to be assessed from a legal point of view.

\begin{quote}\emph{RQ1: What is the definition of identity fraud and which concrete problems can be identified?}\end{quote}
The definition of identity fraud is best described by De Vries \etal as "Identity Fraud is to obtain, to possess or to create intentionally, (and) (unlawfully or without consent) false means of identification in order to commit unlawful behaviour, or to have the intention to commit unlawful behaviour." \cite{97408536fd1c4f4e9d1615b7a4a4473e}. For this research profiling of Formal functional identity data is relevant. 'Physical and bio-metric identity data' and 'Means of identity data' are out of scope because these (sub)domains have their own expertise, rules and exceptions. 
In the design stage of this research the definition of identity fraud was considered concrete enough to start designing solutions. However, during the research it took more time to define and scope identity fraud to define operational problems. Future research could cover a quantified method to assess impact on identity fraud when a certain solution is implemented. Possibly resulting in conclusions containing correlation and no causality because proving an event not happening would also assume one could have prevented it already leading to a paradoxical conclusion.\pagebreak

\begin{quote}\emph{RQ2: Which requirements are significantly relevant to assess technical suitability and trade offs of these technologies for broader usage within the Dutch government?}\end{quote}
This research question is answered in Section \ref{s:overview} by providing an utility tree which contains the five main relevant Quality attributes. Namely, Confidentiality, Reusability, Maturity, Authenticity and Accountability. In practice, preventing identity fraud is only one of the many business goals to take into account. Also, it could depend on the area of expertise of the beholder how to answer this question. Confidentiality is an Architecturally Significant Requirement and a Quality attribute of ISO 25010, but not the only Quality attribute that is relevant to consider when designing a system. Also, one could argue it's not quantifiable what kind of real world problems are mitigated with each implemented pattern of tactic in Chapter \ref{s:Implementation}. However, logical reasoning should provide grip evaluating or designing a system. For future work it can be relevant to investigate the impact of mitigating measures and what will drive a citizen to use, or not use, a government system. Providing and replicating a set of data should be considered a viable solution in the historical perspective of systems based on batch processing and limitations on bandwidth. Duplication of data and overhead in processing are not mentioned as concerns by interviewees or consulted experts. However, it could be considered to explicitly query on these quality attributes to take in consideration from a sustainability point of view.

\begin{quote}\emph{RQ3: Which architectural tactics and patterns are available and can be used to mitigate identity fraud?}\end{quote}
The selected patterns and tactics, Zero-knowledge proof, data minimization, pseudonymization, authentication and encryption are based on the input of both participated interviewee experts whose transcribed semi-structured interviews are available and consulted experts who have provided pointers to operational problems and available literature. Authentication and encryption can be considered as an obligated set of tactics, because the NORA mentions them as mandatory techniques to apply \cite{NORA_PasToeOfLegUit}. Zero-knowledge proof can be considered as a possible patterns but needs to be evalueted per use-case. While tactics as Data minimization and pseudonymization are a set who can be applied broadly. Two of the concerns not solved in Chapter \ref{s:Implementation} are C-01 "Knowledge of systems an what is legally allowed" and C-02 "Knowledge on encryption and standards is scarce". Both concerns refer to a knowledge problem which could result in problems during implementation or life-cycle. Section \ref{Implications} will state more explicitly  Implications for future research on this topic.