\chapter{Introduction}\label{s:intro}

%\todo{
%This section includes some motivations behind the work, explicitly or implicitly
%highlights the research question, provides a high-level explanation of the
%solution, and describes the contributions.}

\section{Motivation}
The objective of this research is combining the problem of identity theft with technical solutions, applied in information systems.Not for a sole purpose or a single software solution, but generically usable patterns and tactics which can be applied building software.

\section{Research question}
How can information systems reduce the possibility to commit identity fraud, without losing the benefit of exchanging personal data of a citizen within the government of The Netherlands, and how can further suitability for these methods be assessed?

\begin{quote}\emph{RQ1: What is the definition of identity fraud and which concrete problems can be identified?}\end{quote}
\begin{quote}\emph{RQ2: Which requirements are significantly relevant to assess technical suitability and trade offs of these technologies for broader usage within the Dutch government?}\end{quote}
\begin{quote}\emph{RQ3: Which available architectural tactics and patterns are available and can be used to mitigate identity fraud?}\end{quote}

\break

\section{Scientific and practical contribution of this research}
Implementing technology means a system should be in place and therefore a software architecture is present. Bass '\etal \cite{Bass2015SoftwareAI} defines software architecture as "a program or computing system is the structure or structures of the system, which comprise software elements, the externally visible properties of those elements, and the relationships among them." Also, Bass \etal formulate every software system has an architecture, only not always formally described.
The research objective is to explicitly describe goals, requirements and concerns that are relevant when designing information systems to mitigate identity fraud. Thereafter, breaking it down this information in a formal description of Quality attributes (QA) and Architectural Significant Requirements (ASR). Based on these QA's and ASRs a selection of available architectural patterns and tactics will be discussed, and possible trade-offs will be assessed. 

This will contribute in selecting the appropriate technologies easier and how to describe which type of identity fraud it can prevent. Making use of a selection of architectural patterns and tactics that already exist will make it possible for practical application. The provided definition and quantification of Identity theft can be used to raise awareness on this problem and serve as an base for future research. The structure of this research can facilitate in technical solutions on other topics.

The contribution to science and practice will be achieved by defining a formal way to assess patterns and tactics that could mitigate identity fraud by implementing technology. Based on the ISO 25010 standard of Software Quality Attributes, not only the attribute of 'Security' will be taken into account. A broader perspective of goals and Architectural Significant Requirements, together with concerns will support a broader view when developing information systems. The patterns and tactics can help designers and builders build and assess their information system.

This research will not answer the question which method should be used for implementation. Also, it can be assumed business goals, concerns, requirements and technology will evolve over time, resulting in a continuously evolving set of variables to take in account.



