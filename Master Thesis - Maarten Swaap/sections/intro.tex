\chapter{Introduction}\label{s:intro}

%\todo{
%This section includes some motivations behind the work, explicitly or implicitly
%highlights the research question, provides a high-level explanation of the
%solution, and describes the contributions.}

\section{Motivation}
Main objective of this research is scoping the problem of identity theft to make it possible to decompose the problem to quality attributes of the ISO 25010 standard \cite{ISO:25010:2011} to support matching process of generically usable patterns and tactics which can be applied building software.

\section{Research question}
How can information systems reduce the possibility to commit identity fraud, without losing the benefit of exchanging personal data of a citizen within the government of The Netherlands, and how can further suitability of these methods be assessed?

\begin{quote}\emph{RQ1: What is the definition of identity fraud and which concrete problems can be identified?}\end{quote}
\begin{quote}\emph{RQ2: Which requirements are significantly relevant to assess technical suitability and trade offs of these technologies for broader usage within the Dutch government?}\end{quote}
\begin{quote}\emph{RQ3: Which available architectural tactics and patterns are available and can be used to mitigate identity fraud?}\end{quote}

\break

\section{Scientific and practical contribution of this research}
Implementing a system implies software architecture is present and architectural choices have been made. Bass '\etal \cite{Bass2015SoftwareAI} defines software architecture as "a program or computing system is the structure or structures of the system, which comprise software elements, the externally visible properties of those elements, and the relationships among them." Also, Bass \etal claim every software system has an architecture. However, the software architecture is not always formally described.
The research objective is to identify goals and concerns relevant when designing information systems which could mitigate identity fraud. Thereafter, combining these goals and concerns in a formal description of Architectural Significant Requirements (ASR) categorized in Quality attributes (QA) based on ISO 25010 \cite{ISO:25010:2011}. Based on these QA's and ASRs a selection of available architectural patterns and tactics will be discussed and possible trade-offs will be reasoned upon.

This will contribute in selecting the appropriate technologies and facilitates mitigating identity fraud. Making use of a selection of already existing architectural patterns and tactics which can be applied in information systems. The provided definition and quantification of Identity theft can be used to raise awareness on this problem and serve as a base for future research on this topic.

Contribution to science and practice will be achieved by defining relevant quality attributes for designing information systems. Concerning the Quality attributes of the ISO 25010 standard not only the attribute 'Security' will be taken into account. Business goals and concerns will provide a broader perspective for developing information systems. The patterns and tactics can help designers and builders build and assess their information system.

This research facilitates in examples and not the definition on which methods should be used for implementation. Also, it can be assumed business goals, concerns, Architectural significant requirements and technology will evolve over time, resulting in a continuously evolving set of variables to take in account when designing or evaluating information systems.