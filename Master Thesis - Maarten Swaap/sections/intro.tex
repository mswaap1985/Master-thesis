\chapter{Introduction}\label{s:intro}

%\todo{
%This section includes some motivations behind the work, explicitly or implicitly
%highlights the research question, provides a high-level explanation of the
%solution, and describes the contributions.}

\section{Motivation}
A report of the Auditdienst Rijk (ADR)\cite{ADR} takes in account numbers of the citizen service for Identity Fraud at the National Office for Identity Data (Rijksdienst voor Identiteitsgegevens). Somewhere around 4.0000 citizens a year contact this service to report identity fraud. In some cases, the impact is low and only a discomfort. In other cases, the fraud results in identity theft that can lead, for example, to debts or fines.\par 
One of the main objectives of the Digitale Overheid (Digital Government) is to 'Protect fundamental rights and public values'\cite{DO_agenda}. One could argue protecting the identity and privacy of citizens is part of this objective.

The objective of this research is to help reduce the opportunity of identity theft and simultaneously improve citizens’ privacy by making usage of encryption technologies. During this research it became clear definitions of both identity fraud and encryption technologies are needed to eventually implement mitigating actions. The scope of this research is to collect a set of uniform needed requirements and conditions plotted on different type of identity fraud. Objective is to implement these technologies easier and how to describe which type of identity fraud it can prevent. Starting with encryption technologies that already exist and making it possible for future research to adapt the outcomes of this research for new technologies.

Because the ADR researched the Burger Service Nummer (BSN) and the role as a Unique Identification Number (UIN) the research started around this number. During the research it became clear that the BSN is part of the identity and only one of the enablers for identity fraud.\par

\section{Research question}
How can encryption technologies reduce the possibility to commit identity fraud, without losing the benefit of exchanging personal data of a citizen within the government of The Netherlands, and how can further suitability for these methods be assessed?

\begin{quote}\emph{RQ1: Which forms of identity fraud can be identified that could be mitigated with usage of encryption methodologies?}\end{quote}
\begin{quote}\emph{RQ2: Which forms of encryption methodologies are available and already implemented or being implemented?}\end{quote}
\begin{quote}\emph{RQ3: Which requirements are significantly relevant to assess technical suitability and trade offs of these technologies for broader usage within the Dutch government?}\end{quote}

\section{Scientific and practical contribution of this research}
Before starting this research, the definition of identity fraud looked straightforward. However, it is needed to define and scope which types of identity fraud exist and how the risks could be mitigated by making use of encryption technologies. To be more precise, this research is scoped on proven encryption technology, that is already in use within the Dutch government and has been peer reviewed. 

The contribution to science and practice will be achieved by defining a basic framework that takes in account types of identity fraud and on the other hand how they can be mitigated by implemented encryption technologies. Encryption is associated with security and privacy, but these are only part of software architecture and implementations. Therefore, this framework will take in account more Software Quality attributes from the ISO 25010 standard. Together, with business goals, significant requirements and concerns this can be used generically when developing new software. 

This research will not answer the question which method should be used for implementation. The framework can help designers and builders of software systems assessing suitability. Also, it can be assumed Identity fraud and requirements will evolve over time, resulting in a continuously evolving framework.
