% this file is called up by thesis.tex
% content in this file will be fed into the main document

%: ----------------------- name of chapter  -------------------------
\appendix 
\chapter{Interviewees} \label{Appendix A} % top level followed by section, subsection 
This section will describe the role of interviewees. Appendix B and C contain a cross-reference with this interviewees. 

\begin{longtblr}[
  caption = {List of interviewees and purpose for consultation},
  label = {tab:interviewees},
]{
  colspec = {|XX[2]|},
  rowhead = 1,
  hlines,
  row{even} = {gray9},
  row{1} = {red7},
} 
Interviewee number and role name & Purpose for consultation\\
 I-01 Stelselspecialist (EN: System specialist)   &  \\
 I-02 Informatie analist (EN: Information Analyst) &  \\
 I-03 Regie architect (EN: directing architect) & \\
 I-04 Adviseur Digitalisering (EN: Advisor digitization) &\\
 I-05 Chief Information Security Officer & Consulted as a source for relevant standards and techniques (patterns and tactics)\\
\end{longtblr}

\chapter{Business goals} \label{Appendix B} % top level followed by section, subsection
This section will cover output of interviews and contains a brief overview of business goals. Raw output of interviews and code document (as far as NDA allows \todo{check NDA within RvIG}) are added in the replication package of this research.


\begin{longtblr}[
  caption = {List of Business Goals},
  label = {tab:business_goals},
]{
  colspec = {|XX[3]|},
  rowhead = 1,
  hlines,
  row{even} = {gray9},
  row{1} = {red7},
} 
Business goal number and name & Description & Addressed by\\
 BG-01 Explainable solutions   &   Communicate simple how technology works to raise awareness and support. Also, provide more detailed scientific information for experts to contribute as a community. &  I-03\\
 BG-02 (Re-)Use of available standards and patterns &  Reusage of standards and patterns is a goal to keep systems maintainable and portable. It also helps in building a knowledge base that can be applied on multiple products & I-03\\
 BG-03 Apply a 'Zero knowledge proof' or minimalize data exchange & "Minimalize the amount of data transfered. If data is not available it can not be stolen in case of for example a databreach.GDPR states a minimalization of data in Article 5. Only provide information that is needed for the initial purpose.Ultimately, it's desired to only acknowledge if the provided personal information is correct, without providing the information itself. Also known as a 'Zero knowledge proof'." & I-01 and I-03 \\
BG-04 Maintain a unique identity of a person within a whole chain & Big advantage of maintaining a unique identity (number) of a person within a whole chain of government body, prevent rework and multiple different ways of administering a single person. & I-01 and I-02 \\
 BG-05 Provide ease of use and uniform (self)service solutions to a citizen, together with chain partners &  Services provided to citizens (both residential and non-residential) are uniformly provided by each government body. Currently, this is mainly provided by having a DigiD that is on itself based on the unique BSN of a citizen. & I-01 and I-02\\
 BG-06 Provide identity data outside government organisations, while preserving citizens privacy & Usage of identity information of the 'Basisregistratie personen (BRP)' is now limited to a small selection of government and non-government organizations. It can be assumed more non-government organizations will need to verify an identity, while the identity information itself is not provided to that company. & I-04 \\
\end{longtblr}

\chapter{Concerns} \label{Appendix C}
This section will cover output of interviews and contains a brief overview of concerns. Raw output of interviews and code document are added in the replication package.

\begin{longtblr}[
  caption = {List of Concerns},
  label = {tab:concerns},
]{
  colspec = {|XX[3]|},
  rowhead = 1,
  hlines,
  row{even} = {gray9},
  row{1} = {red7},
} 
Concern number and name & Description & Addressed by\\
C-01 Knowledge of systems and what is legally allowed    &   Area of expertise within the identity domain is complex and system specialists (NL: Stelselspecialisten) and other specialists are a rare resource. History of law and systems is implicit knowledge or hard to find in documentation. When designing new systems it's needed to comprehend the domain and its internal and external influences. & I-03\\
 C-02 Knowledge of encryption and standards is scarce &  Area of expertise within technological field of encryption standards is complex and specialists are a rare resource. & I-03\\
 C-03 Secure transfer of personal data  &  Transfer of personal data is necessary by law, but needs to be done in a safe and secure way. & I-03 \\
C-04 Uniqueness of key, but vulnerable because of process
& One BSN can be assigned to multiple persons or multiple BSN can be assigned to one person. These incidents can happen and are mitigated by processes within RvIG and can be reported by the citizen at the Meldpunt Fouten in Overheidsregistraties since January 2021. An suggested alternative could be the A-nummer, which' uniqueness is guaranteed in by referential intergrety of the database. & I-01 and I-02\\
C-05 Uniqueness of key, but vulnerable because of misuse & BSN as a UIN is broadly used within government bodies and commercial parties. It's not possible to control all applications of BSN, while this UIN is important to identify a unique person. Therefore, it's a vulnerable object within identity data. & I-01 \\
C-06 Identity misuse & Is the person using the identity data the person that belongs to that specific identity & I-02\\
C-07 Citizen sees the government in a broader perspective than the government itself & Usage of BSN is positioned as a number that can only be used by authorized organization, allowed by law. While a citizen sees the (responsibilities of the) government bigger than the government itself. Not allowing organizations to access data of the citizen, while the citizen himself expects the organization to access that data could have a mismatch in service expectations. & I-01\\
C-08 BSN not available, while rights may apply & A person lives and/or works in The Netherlands, but does not have a BSN. Or a person lives in a foreign country, but has rights in The Netherlands (like a pension). In that circumstances there are rights, but no identification number & I-01\\
C-09 As a government it’s needed to take in account all possible exceptions & A commercial organization could argue a data quality of for example 98\% is almost perfect and enough to maintain a high level of service. A government body needs to take in account that 2\% of it's population is a few 100.000 people that could have not overseen troubles in their contact with government and non-government organizations that for example authenticate that person and base decisions on the provided data. & I-01\\
C-10 Recognizing and authentication of a person is linked intensively with the BSN & There are processes relying on solely a person having a BSN number, while it could be possible to have rights without having a BSN. Also, authenticating a person is relying a lot on having the correct BSN, while it was not the purpose of BSN to authenticate a person. & I-01 \\
C-11 Laws and practical implementations focus on BSN as a primary key that needs to be used & While the practical implementation could be done without BSN, sometimes it's designated by law this primary key is leading and needs to be used & I-04 \\
C-12 Non-government sectors need to exchange information on citizens & Non-government sectors, like banks, insurance companies or mortgage lenders, need to exchange information of citizens with other non-government organizations. They need to verify an identity before exchanging information, while preserving privacy could make this more complex. Administering signals of fraud or creditworthiness wrongly could impact a citizen negatively. & I-04

\end{longtblr}


% ---------------------------------------------------------------------------
%: ----------------------- end of thesis sub-document ------------------------
% ---------------------------------------------------------------------------

