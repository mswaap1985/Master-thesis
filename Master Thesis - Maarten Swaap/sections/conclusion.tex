\chapter{Conclusion}\label{s:conclusion}
\section{Summary of contributions}
This research defines and quantifies a scope of identity fraud that could impact citizens privacy (1). It demonstrates which Quality Attributes could be considered Architecturally Significant Requirements, not only with a scope on Security (2). Applies practical use-cases with technologies to possibly implement and set an example to apply to future processes (3). The solutions in this research are not decomposed in detail, meaning further refinement is needed before building and implementing. However, the decomposition of Architecturally Significant Requirements can be used for fundamental blocks designing information systems which exchange personal information. 

\section{Implications for future research}\label{Implications}
On terms of knowledge assurance concerns C-01 and C-02 are describing knowledge of systems and legal applications. These concerns are not resolved in this research but need to be addressed and can be input for further future research on embedding knowledge. Not only looking at the aspects of software systems (for example pattern libraries) but also the organizational aspects of embedding knowledge (for example in-sourcing strategies). 
Also, it needs to be noted not only information technology needs to play a part in solutions. Further research could address other areas of expertise. For example communication and psychology to raise awareness and explain behaviour. Or expertise of law to adapt laws to restrict or oblige certain ways of exchanging personal information. 
Section \ref{s:related} related work shows methods on evaluation and comparison when selecting appropriate software architectures. For future research these methods could be applied on outcomes of this research.

%\todo{
%Briefly summarize your contributions, and share a glimpse of the implications of
%this work for future research.
%}