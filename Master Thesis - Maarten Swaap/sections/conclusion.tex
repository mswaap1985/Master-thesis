\chapter{Conclusion}\label{s:conclusion}
\section{Summary of contributions}
This research defines and quantifies a scope of identity fraud which could impact citizens privacy (1). It demonstrates which Quality Attributes are considered to be Architecturally Significant Requirements not only focusing on the Quality attribute of Security (2). Applies practical use-cases with technologies to possibly implement and set an example to apply to future processes (3). The solutions in this research are not decomposed in detail, meaning further refinement is needed before building and implementing. However, the decomposition of Architecturally Significant Requirements can be used for fundamental blocks designing and evaluating information systems which exchange personal information. 

\section{Implications for future research}\label{Implications}
On terms of knowledge assurance concerns C-01 and C-02 are describing knowledge of both systems and legal applications. These concerns are addressed but more research is needed on embedding knowledge not only focusing on the aspects of software systems (for example pattern libraries) but also the organizational aspects of embedding knowledge (for example in-sourcing strategies of personnel). 
Also, it needs to be noted not only information technology needs to play a part in solutions. Further research could address other areas of expertise. For example communication and psychology to raise awareness and explain behaviour of decision makers and users. Or expertise of law to adapt laws to restrict or oblige certain ways of exchanging personal information. Section \ref{s:related} related work shows methods on evaluation and comparison when selecting appropriate software architectures. For future research these methods could be applied on outcomes of this research.

%\todo{
%Briefly summarize your contributions, and share a glimpse of the implications of
%this work for future research.
%}