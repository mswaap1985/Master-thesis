\chapter{Background}\label{s:background}
%\todo{This section provides the necessary context to help the reader understand the remainder of the thesis.}
\section{Problem definition}
The Unique Identity Number (UIN) of the Dutch citizen. The ‘Burger Service Nummer’, abbreviated by BSN (English: ‘Citizen Service Number’). It’s comparable with the Social Security Number in the USA, but it is also being used for different purposes in The Netherlands. \par
The Auditdienst Rijk (ADR)\cite{ADR} of the Dutch government defined gaps and possible improvements in usage of this number. In this report the big advantage of BSN is concluded to be a unique identifier for many different government and non-government organizations. Making their administration done more easily. On the other hand, the main objective for BSN was to prevent identity fraud. Facts and figures in the report of ADR are showing Identity Fraud is becoming a bigger issue. The problem for citizens is that using the BSN could impact the privacy of citizens negatively, while having great benefits for the Dutch government. The same BSN is re-used for different purposes. For example, a bank uses the same BSN of a citizen as a hospital. If the hospital would have a data breach, the BSN could be re-used for committing identity fraud at a bank. By a different pseudonymized or tokenized UIN for different purposes, the dataset of a person becomes compartmentalized. It doesn’t mean it’s impossible to combine datasets, but without an identical primary key it’s more difficult. 

\subsection{Definition, pros and cons of a UIN}
The World Bank \cite{WorldBank_UIN} defines ID numbers "In the context of foundational systems, ID numbers are considered to be “unique when: firstly, the number-generating process ensures that no two people within the system share the same number and secondly, a deduplication process ensures that the same person does not have multiple identity records or numbers (i.e., that they are unique in the database)." The World Bank identifies two derived strengths, namely ‘Uniqueness and deduplication’ and ‘Data matching and interoperability’. This means each person can be identified uniquely, allowing for a fast and easy exchange of identity related information between different organizations.\par 
However, the benefit of fast and easy exchange of identity related information, is also highly susceptible to misuse. Identity Fraud and Theft leans heavily on the benefit of ‘Data matching and interoperability’ of the UIN. A second form of misuse that leans on this benefit is Unauthorized data correlation. \par

\subsection{Definition of Identity Fraud}
While executing this research it became clear that identity fraud is a catch-all term. De Vries \etal \cite{97408536fd1c4f4e9d1615b7a4a4473e} have analysed 30 international definitions and formulated a general description "Identity fraud is to obtain, to possess or to create intentionally, (and) (unlawfully or without consent) false means of identification
in order to commit unlawful behaviour, or to have the intention to commit unlawful behaviour." In the footnote it states: "It must be noted that ‘false’ in the description refers to the idea that the means of identification do not identify the person who uses them truthfully."
This research will make use It's needed to say this is a generic definition. 

\subsection{Quantitative impact of Identity Fraud}
The research of De Vries \etal states it's difficult to quantify identity fraud, because it's incorporated in other crimes {\cite{Vries2007IdentiteitsfraudeEA}}{\todo{toevoegen bronvermelding}}. Quantification is relevant for this research, because when a form of identity fraud can be quantified, the mitigation of this form can be argued. Emphasis of this research will be argumentation and not quantification. This research uses quantification as an input, but does not conduct research on quantification of output. {\todo{Opnemen in future research}}

However, over time Statistics Netherlands (CBS) quantified data of identity fraud, based on statistical data. Purely looking at the data classified by Identity Fraud, CBS defines it: "Without permission, via internet, making usage of someones personal data for financial gain, for example by withdraw or transfer of money, take out a loan or request of official documents." Roughly 0.5\% of Dutch citizens in 2019 became victim of identity fraud. Based on a population of 17 million, this means roughly 86.000 people are confronted with Identity Fraud each year. 
{\todo{TABEL HIER https://opendata.cbs.nl/#/CBS/nl/dataset/82464NED/table?ts=1637054627124
https://opendata.cbs.nl/#/CBS/nl/dataset/82464NED/barh?ts=1637054521497
https://opendata.cbs.nl/#/CBS/nl/dataset/82464NED/table?ts=1637054352194}}.
When looking at a broader scope, to anticipate on a definition creep, another research of the CBS is relevant {\cite{CBS_casualtiesDigitalCrime}} and zooms in on the way casualties act.

\subsection{Legal implications}
Dutch laws enforce the exchange of identity data of a citizen and explicitly state to include BSN in this dataset. These laws exist in order to stimulate controlled exchange of personal data. This applies to government bodies , pension providers  and healthcare organizations. Usage of an alternative method, not explicitly providing a BSN, can be interpreted as not complying with these laws. This interpretation is not in scope for my thesis research.\par
However, technologies like encryption and pseudonymisation are explicitely stated in General Data Protection Regulation (GDPR, REGULATION (EU) 2016/679) {\cite{GDPR}} to be in place as a safeguard. 
Other rules and regulations may apply, but are not in scope.

\subsection{Pseudonymization or tokenization of a UIN}
Researching possible solutions in literature and consulting experts showed two viable methods worth analyzing. Pseudonymization and tokenization. Pseudonymization of BSN already has been applied within the Dutch government by Logius. Previous work by AuditDienst Rijk on the usage of BSN  already discussed pseudonymization as an option. Tokenization is an alternative method largely applied in payment sector (Adyen and Apple Pay) to ensure privacy of customers. This method is suggested by the World Bank as a possible solution to replace a UIN, because it’s been applied within the governments of Austria, India and Estionia.\par
The difference between these two techniques is mainly the method of reversing. While pseudonymization needs a key to decrypt, the techniques behind tokenization need a form of ledger or table to lookup the original record.
Off course, there are a lot more technological methods. However, the scope of this research will not be assessing all of them, rather creating a useful framework to assess these methods based on requirements.\par
\textbf{Pseudonymization} 
GDPR \cite{GDPR} defines ‘pseudonymisation’: means the processing of personal data in such a manner that the personal data can no longer be attributed to a specific data subject without the use of additional information, provided that such additional information is kept separately and is subject to technical and organisational measures to ensure that the personal data are not attributed to an identified or identifiable natural person; \par
\textbf{Tokenization} is described by The World Bank  as “Tokenization can protect privacy by ensuring that only tokens, rather than a permanent identity number or other UIN, are exposed or stored during a transaction.” Also, by this definition, the same person is represented by different tokens in different databases. A fundamental property, besides it’s unique character, is it’s not possible to reverse engineer a person’s identity, because the unique token does not contain this data.\par
The World Bank defines two primary types of tokenization. Firstly, \textbf{Front-end tokenization} is the creation of a token by the user as part of an online service that can later be used in digital transactions in place of the original identifier value. Secondly, in case of \textbf{Back-end tokenization} the identity provider (or token provider) tokenizes identifiers before they are shared with other systems, limiting the propagation of the original identifier and controlling the correlation of data. Back-end tokenization is done automatically by the system without user intervention, meaning that people do not need to do anything manually or understand why they would need to create tokens, eliminating any potential digital divide and protecting identifiers and UIN at source.”